\documentclass[a4paper]{article}

\usepackage[utf8]{inputenc}
\usepackage[T1]{fontenc}
\usepackage[spanish]{babel}
\usepackage[margin={20mm,25mm}]{geometry}

\usepackage{ifthen}
\usepackage{mathtools}
\usepackage{xcolor}
\usepackage{booktabs}
\usepackage{circuitikz}
\usepackage{fancyhdr}
\usepackage{parskip}
\usepackage{hyperref}
\usepackage{graphicx}

% --------------------
% DATOS DEL INFORME
\newcommand{\numeroTarea}   {X}                 % <-- Reemplazar X con el número de la tarea
\newcommand{\numeroGrupo}   {N}                 % <-- Reemplazar N con el número de su grupo
\newcommand{\nombrePrimero} {Nombre Primero}    % <-- Reemplazar con el nombre del primer integrante
\newcommand{\rolPrimero}    {12345678-9}        % <-- Reemplazar con el rol del primer integrante
\newcommand{\nombreSegundo} {Nombre Segundo}    % <-- Reemplazar con el nombre del segundo integrante
\newcommand{\rolSegundo}    {12345678-9}        % <-- Reemplazar con el rol del segundo integrante
% --------------------

% --------------------
% NO MODIFICAR ESTA PARTE
\title{Informe Tarea \numeroTarea \\ \large
    \ifthenelse{\equal{\numeroTarea}{1}}{Circuito Combinacional}{}
    \ifthenelse{\equal{\numeroTarea}{2}}{Circuito Secuencial}{}
    \ifthenelse{\equal{\numeroTarea}{3}}{Lenguajes de Descripción de Hardware}{}
    \ifthenelse{\equal{\numeroTarea}{4}}{ARM Assembly}{}
    \ifthenelse{\equal{\numeroTarea}{X}}{Tema de la tarea}{}
}
\author{\textbf{Grupo \numeroGrupo} \\ \begin{tabular}{r @{\quad} l}
    \nombrePrimero & \rolPrimero \\
    \nombreSegundo & \rolSegundo
\end{tabular}}
\date{\today}

\setlength{\parindent}{15pt}
\addto\captionsspanish{\renewcommand{\tablename}{Tabla}}
% --------------------

\begin{document}

\begin{titlepage}
    \maketitle
    \thispagestyle{empty}
    
    \begin{abstract}
        {\color{red} \textbf{Antes de entregar el informe deben eliminar todas las secciones en rojo, ya sea borrando el contenido o comentando las líneas usando un signo de porcentaje (\%).}
    
        Acá deberán escribir un resumen de su informe. Debe mencionar todos los puntos importantes del desarrollo de la tarea e idealmente tener un máximo de 10 líneas, similar al \textit{Abstract} de un paper.}
    \end{abstract}
    
    \vfill
    \tableofcontents
\end{titlepage}

\section{Desarrollo de la tarea}
{\color{red} Acá deben incluir el desarrollo completo y en detalle de la tarea. Recuerden mencionar cualquier ecuación o fórmula que utilicen, e incluir figuras y tablas para mostrar datos o resultados. Más abajo encontrarán un ejemplo de tabla centrada (\ref{tab:ej_tabla}), un ejemplo de figura centrada (\ref{fig:ej_figura}), y un ejemplo de ecuación (\ref{eq:ej_ecuacion}).

Si quieren referenciar una ecuación, figura o tabla deben usar la sintaxis \texttt{\textbackslash ref\{label\}}, donde \texttt{label} es el identificador que le asignaron a su referencia. Por ejemplo, pueden ver extractos de código para formar una tabla, una figura y una ecuación en las figuras \ref{fig:v_tabla}, \ref{fig:v_figura} y \ref{fig:v_ecuacion} respectivamente.}

% --------------------
% EJEMPLOS DE FIGURAS, TABLAS Y ECUACIONES
\begin{table}[!htbp]
    \centering % Este comando centra todo dentro del ambiente table
    \begin{tabular}{l c r} \toprule % Alineación de cada columna: l a la izquieda, c al centro, y r a la derecha
        \textbf{Columna a la izquierda} & \textbf{otra al centro} & \textbf{y una a la derecha} \\ \midrule
        Aquí hay texto en la tabla & \multicolumn{2}{l}{... también cubriendo dos columnas usando \texttt{\textbackslash multicolumn}} \\
        10 & 10 & 10 \\
        \multicolumn{3}{c}{Los números de arriba están alineados a la izquierda, centro y derecha respectivamente} \\ \bottomrule
    \end{tabular}
    \caption{Este es el título de la tabla, puede ser lo que quieran}
    \label{tab:ej_tabla} % Este es el identificador de la tabla
\end{table}

\begin{figure}[!htbp]
    \centering % Este comando centra todo dentro del ambiente figure
    \includegraphics[width=0.6\textwidth]{logo_usm.png} % Con esto pueden incluir imágenes que hayan subido a Overleaf
    \caption{Este es el título de la figura, en este caso es el logo de la universidad}
    \label{fig:ej_figura} % Este es el identificador de la figura
\end{figure}

\begin{equation}
    \overline{A B} = \overline{A} + \overline{B}
    \label{eq:ej_ecuacion} % Este es el identificador de la ecuación
\end{equation}

\begin{figure}[!htbp]
    \centering
    \begin{minipage}{0.7\textwidth}
        \begin{verbatim}
            \begin{table}[!htbp]
                \centering
                \begin{tabular}{columnas} \toprule
                    heading 1 & heading 2 \\ \midrule
                    contenido & texto \\
                    números & etc. \\ \bottomrule
                \end{tabular}
                \caption{Título de la tabla}
                \label{tab:identificador}
            \end{table}
        \end{verbatim}
    \end{minipage}
    \caption{Extracto de código para incluir una tabla}
    \label{fig:v_tabla}
\end{figure}

\begin{figure}[!htbp]
    \centering
    \begin{minipage}{0.7\textwidth}
        \begin{verbatim}
            \begin{figure}[!htbp]
                \centering
                \includegraphics[opciones]{imagen.png}
                \caption{Título de la figura}
                \label{fig:identificador}
            \end{figure}
        \end{verbatim}
    \end{minipage}
    \caption{Extracto de código para incluir una figura}
    \label{fig:v_figura}
\end{figure}

\begin{figure}[!htbp]
    \centering
    \begin{minipage}{0.7\textwidth}
        \begin{verbatim}
            \begin{equation}
                a + b = c
                \label{eq:identificador}
            \end{equation}
        \end{verbatim}
    \end{minipage}
    \caption{Extracto de código para incluir una ecuación}
    \label{fig:v_ecuacion}
\end{figure}
% --------------------

\section{Resultados y análisis}
{\color{red} Acá deben presentar los resultados que obtuvieron al probar su tarea, tanto con los casos de prueba propuestos en el enunciado como con casos que diseñen ustedes mismos. Además deben analizar los resultados mencionando si son correctos y si hay discrepancias con lo que debiera ser.}

\section{Conclusiones}
{\color{red} Por último acá deben escribir una conclusión respecto a la finalización de la tarea y el nivel de completitud que alcanzaron, en caso de no terminarla. Deben también incluir una reflexión personal respecto a la tarea, los problemas que tuvieron, y qué podrían mejorar en caso de tener que trabajar nuevamente con el contenido de la tarea.}

\end{document}
